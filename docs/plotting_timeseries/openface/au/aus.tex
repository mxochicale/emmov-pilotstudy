\documentclass[a4paper,12pt]{article}
%% Language and font encodings
\usepackage[english]{babel}
\usepackage[utf8x]{inputenc}


%% Sets page size and margins
%\usepackage[a4paper,top=1cm,bottom=4mm,left=3cm,right=0mm,marginparwidth=1.75cm]{geometry}
%\usepackage[a4paper,top=0cm,bottom=3mm,left=3mm,right=0mm,marginparwidth=0cm]{geometry}
\usepackage{geometry} % for \newgeometry{} and \restoregeometry

%% Useful packages
\usepackage{amsmath}
\usepackage{graphicx}
\usepackage[colorinlistoftodos]{todonotes}
\usepackage[colorlinks=true, allcolors=blue]{hyperref}
\usepackage{url}%for \path{}
\usepackage{lscape}
\pagenumbering{gobble}%no numbers in the webpage

%\usepackage[counterclockwise]{rotating} %sidewaysfigure





\graphicspath{{/home/map479/mxochicale/github/DataSets/emmov/plots_timeseries/}}
%\title{ OpenFace Time Series }

\title{Facial Action Units}
\author{Miguel P Xochicale \\
% map479@bham.ac.uk \\
School of Engineering\\
%Department of Electronic, Electric and System Engineering\\
University of Birmingham, UK}


\begin{document}


\newgeometry{top=20mm,bottom=20mm,left=20mm,right=20mm}
\maketitle

\begin{abstract}
Time series for Facial Action Units. 
\end{abstract}


\section{Description}

For the following plots, there are two type of face action units: 
AU\_c for presence (0 not presence, 1 presence) and 
AU\_r for intensity (0 not presence, 1 presence at minimum intensity and 5 present at maximum intensity.)
Baltrusaitis et al. \cite{baltrusaitis2016} mentioned 
that the AU predictions would not be always consistence because 
"the predictors have been trained separately and in different datasets".



\section{r-scripts and data paths}
\path{~/github/emmov-pilotstudy/code/r-scripts/plotting} \\
\path{timeseries-openface.R}
is used to generate the plots which are saved at: \\
\path{~/github/DataSets/emmov/plots_timeseries/*}



The data is available in \cite{mxochicale2018}.



\bibliographystyle{apalike}
\bibliography{references}

%\newpage

%\newgeometry{top=0cm,bottom=3mm,left=3mm,right=0mm,marginparwidth=0cm}
%\restoregeometry




\begin{figure}
\centering
\includegraphics{openface-timeseries_AU01_cr}
\caption{AU01: Inner brow raiser}
\end{figure}

\begin{figure}
\centering
\includegraphics{openface-timeseries_AU02_cr}
\caption{AU02: Outer brow raiser}
\end{figure}

\begin{figure}
\centering
\includegraphics{openface-timeseries_AU04_cr}
\caption{AU04: Brow lowerer}
\end{figure}

\begin{figure}
\centering
\includegraphics{openface-timeseries_AU05_cr}
\caption{AU05: Upper lid raiser}
\end{figure}


\begin{figure}
\centering
\includegraphics{openface-timeseries_AU06_cr}
\caption{AU06: Check raiser}
\end{figure}

\begin{figure}
\centering
\includegraphics{openface-timeseries_AU07_cr}
\caption{AU07: Lid tightener }
\end{figure}

\begin{figure}
\centering
\includegraphics{openface-timeseries_AU09_cr}
\caption{AU09: Nose wrinkler}
\end{figure}

\begin{figure}
\centering
\includegraphics{openface-timeseries_AU10_cr}
\caption{AU10: Upper lid raiser}
\end{figure}

\begin{figure}
\centering
\includegraphics{openface-timeseries_AU12_cr}
\caption{AU12: Lid corner puller}
\end{figure}


\begin{figure}
\centering
\includegraphics{openface-timeseries_AU14_cr}
\caption{AU14: Dimpler}
\end{figure}

\begin{figure}
\centering
\includegraphics{openface-timeseries_AU15_cr}
\caption{AU15: Lid corner depressor}
\end{figure}


\begin{figure}
\centering
\includegraphics{openface-timeseries_AU17_cr}
\caption{AU17: Chin raiser}
\end{figure}

\begin{figure}
\centering
\includegraphics{openface-timeseries_AU20_cr}
\caption{AU20: Lid stretched}
\end{figure}


\begin{figure}
\centering
\includegraphics{openface-timeseries_AU23_cr}
\caption{AU23: Lid tightener}
\end{figure}

\begin{figure}
\centering
\includegraphics{openface-timeseries_AU25_cr}
\caption{AU25: Lips part}
\end{figure}


\begin{figure}
\centering
\includegraphics{openface-timeseries_AU26_cr}
\caption{AU26: Jaw drop}
\end{figure}


\begin{figure}
\centering
\includegraphics{openface-timeseries_AU45_cr}
\caption{AU45: Blink}
\end{figure}


\begin{figure}
\centering
\includegraphics{openface-timeseries_AU_intensity}
\caption{AU intensity.}
\end{figure}

\begin{figure}
\centering
\includegraphics{openface-timeseries_AU_presence}
\caption{AU presence.}
\end{figure}





%\begin{figure}
%\centering
%\includegraphics{openface-timeseries_X_all}
%\caption{Landmarks in 3D for $X$ variable.}
%\end{figure}
%
%\begin{figure}
%\centering
%\includegraphics{openface-timeseries_Y_all}
%\caption{Landmarks in 3D for $Y$ variable.}
%\end{figure}
%
%\begin{figure}
%\centering
%\includegraphics{openface-timeseries_Z_all}
%\caption{Landmarks in 3D for $Z$ variable.}
%\end{figure}
%
%



















\end{document}
