\documentclass[a4paper,12pt]{article}
%% Language and font encodings
\usepackage[english]{babel}
\usepackage[utf8x]{inputenc}


%% Sets page size and margins
%\usepackage[a4paper,top=1cm,bottom=4mm,left=3cm,right=0mm,marginparwidth=1.75cm]{geometry}
%\usepackage[a4paper,top=0cm,bottom=3mm,left=3mm,right=0mm,marginparwidth=0cm]{geometry}
\usepackage{geometry} % for \newgeometry{} and \restoregeometry

%% Useful packages
\usepackage{amsmath}
\usepackage{graphicx}
\usepackage[colorinlistoftodos]{todonotes}
\usepackage[colorlinks=true, allcolors=blue]{hyperref}
\usepackage{url}%for \path{}
\usepackage{lscape}
\pagenumbering{gobble}%no numbers in the webpage

%\usepackage[counterclockwise]{rotating} %sidewaysfigure





\graphicspath{{/home/map479/mxochicale/github/DataSets/emmov/plots_timeseries/}}

\title{Gaze and Pose Estimation}
\author{Miguel P Xochicale \\
School of Engineering\\
University of Birmingham, UK}


\begin{document}


\newgeometry{top=20mm,bottom=20mm,left=20mm,right=20mm}
\maketitle

%\begin{abstract}
%
%\end{abstract}


\section{Description}


confidence how confident is the tracker in current landmark detection estimage.

success is the track successful (is there a face in the frame or do we think we tracked it well).

gaze\_0\_x, gaze\_0\_y, gaze\_0\_z Eye gaze direction vector in world coordinates for eye 0 (normalized), eye 0 is the leftmost eye in the image.

gaze\_1\_x, gaze\_1\_y, gaze\_1\_z Eye gaze direction vector in world coordinates for eye 1 (normalized), eye 1 is the rightmost eye in the image.


pose\_Tx, pose\_Ty, pose\_Tz the location of the head with respect to camera in millimetre (positive Z is away from the camera)

pose\_Rx, pose\_Ry, pose\_Rz Rotation is in radians around X,Y,Z axes with the convention R = Rx * Ry * Rz, left-handed positive sign. This can be seen as pitch (Rx), yaw (Ry), and roll (Rz). The rotation is in world coordinates with camera being the origin.

\cite{baltrusaitis2016}.


\section{r-scripts and data paths}
\path{~/github/emmov-pilotstudy/code/r-scripts/plotting} \\
\path{timeseries-openface.R}
is used to generate the plots which are saved at: \\
\path{~/github/DataSets/emmov/plots_timeseries/*}



The data is available in \cite{mxochicale2018}.



\bibliographystyle{apalike}
\bibliography{references}

%\newpage

%\newgeometry{top=0cm,bottom=3mm,left=3mm,right=0mm,marginparwidth=0cm}
%\restoregeometry




\begin{figure}
\centering
\includegraphics{openface-timeseries_confidence_success}
\caption{Cofidence and success variables}
\end{figure}

\begin{figure}
\centering
\includegraphics{openface-timeseries_gaze_0_xyz}
\caption{Eye gaze direction vector in the world coordinates for eye 0.}
\end{figure}

\begin{figure}
\centering
\includegraphics{openface-timeseries_gaze_1_xyz}
\caption{Eye gaze direction vector in the world coordinates for eye 1.}
\end{figure}

\begin{figure}
\centering
\includegraphics{openface-timeseries_pose_RxRyRz}
\caption{Rotation in radians. This can be seen as pitch (Rx), yaw (Ry), and roll (Rz).}
\end{figure}

\begin{figure}
\centering
\includegraphics{openface-timeseries_pose_TxTy}
\caption{Pose Tx and Ty location of the head with respect to camera in milimetre.}
\end{figure}

\begin{figure}
\centering
\includegraphics{openface-timeseries_pose_Tz}
\caption{Pose Tz location of the head with respect to camera in milimetre.}
\end{figure}


%
\end{document}
