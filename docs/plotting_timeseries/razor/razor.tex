\documentclass[a4paper,12pt]{article}
%% Language and font encodings
\usepackage[english]{babel}
\usepackage[utf8x]{inputenc}


%% Sets page size and margins
%\usepackage[a4paper,top=1cm,bottom=4mm,left=3cm,right=0mm,marginparwidth=1.75cm]{geometry}
%\usepackage[a4paper,top=0cm,bottom=3mm,left=3mm,right=0mm,marginparwidth=0cm]{geometry}
\usepackage{geometry} % for \newgeometry{} and \restoregeometry

%% Useful packages
\usepackage{amsmath}
\usepackage{graphicx}
\usepackage[colorinlistoftodos]{todonotes}
\usepackage[colorlinks=true, allcolors=blue]{hyperref}
\usepackage{url}%for \path{}
\usepackage{lscape}
\pagenumbering{gobble}%no numbers in the webpage

%\usepackage[counterclockwise]{rotating} %sidewaysfigure





\graphicspath{{/home/map479/mxochicale/github/DataSets/emmov/plots_timeseries/razor/}}
%\title{ OpenFace Time Series }

\title{Razor Inertial Sensor}
\author{Miguel P Xochicale \\
% map479@bham.ac.uk \\
School of Engineering\\
%Department of Electronic, Electric and System Engineering\\
University of Birmingham, UK}


\begin{document}


\newgeometry{top=20mm,bottom=20mm,left=20mm,right=20mm}
\maketitle

%\begin{abstract}
%Time series for Facial Action Units. 
%\end{abstract}


\section{Description}
Time series for raw data of accelerometer and gyroscope inertial sensor.
The sample rate is 50 Hertz. 


\section{r-scripts and data paths}
\path{~/github/emmov-pilotstudy/code/r-scripts/plotting} \\
\path{timeseries-razor.R}
is used to generate the plots which are saved at: \\
\path{~/github/DataSets/emmov/plots_timeseries/*}

The data is available in \cite{mxochicale2018}.

\bibliographystyle{apalike}
\bibliography{references}

%\newpage

%\newgeometry{top=0cm,bottom=3mm,left=3mm,right=0mm,marginparwidth=0cm}
%\restoregeometry




\begin{figure}
\centering
\includegraphics{razor-timeseries_AccXYZ}
\caption{3D Accelerometer}
\end{figure}

\begin{figure}
\centering
\includegraphics{razor-timeseries_GyroXYZ}
\caption{3D Gyroscope}
\end{figure}











\end{document}
