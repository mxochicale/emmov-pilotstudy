\documentclass[a4paper,12pt]{article}
%% Language and font encodings
\usepackage[english]{babel}
\usepackage[utf8x]{inputenc}


%% Sets page size and margins
%\usepackage[a4paper,top=1cm,bottom=4mm,left=3cm,right=0mm,marginparwidth=1.75cm]{geometry}
%\usepackage[a4paper,top=0cm,bottom=3mm,left=3mm,right=0mm,marginparwidth=0cm]{geometry}
\usepackage{geometry} % for \newgeometry{} and \restoregeometry

%% Useful packages
\usepackage{amsmath}
\usepackage{graphicx}
\usepackage[colorinlistoftodos]{todonotes}
\usepackage[colorlinks=true, allcolors=blue]{hyperref}
\usepackage{url}%for \path{}
\usepackage{lscape}
\pagenumbering{gobble}%no numbers in the webpage

%\usepackage[counterclockwise]{rotating} %sidewaysfigure




\graphicspath{{/home/map479/mxochicale/github/DataSets/emmov/utde/openface/timeseries/}}

\title{OpenFace Time Series for UTDE Analysis}
\author{Miguel P Xochicale \\
School of Engineering\\
University of Birmingham, UK}


\begin{document}


\newgeometry{top=20mm,bottom=20mm,left=20mm,right=20mm}
\maketitle

%\begin{abstract}
%
%\end{abstract}


\section{Description}


Postprocessing techniques (e.g. Savitzky-Golay Filter, Zero Mean Unit Variance)
for OpenFace raw data \cite{baltrusaitis2016}.


\section{r-scripts and data paths}
\path{~/github/emmov-pilotstudy/code/r-scripts/utde/} \\
\path{utde-openface.R}
is used to generate the plots which are saved at: \\
\path{~/github/DataSets/emmov/utde/openface/timeseries/*}
The data is available in \cite{mxochicale2018}.



\bibliographystyle{apalike}
\bibliography{references}

%\newpage

%\newgeometry{top=0cm,bottom=3mm,left=3mm,right=0mm,marginparwidth=0cm}
%\restoregeometry






\begin{figure}
\centering
\includegraphics[width=\textwidth]{pre-of-timeseries_zmuvbase}
\caption{S-G confidence and success.}
\end{figure}

\begin{figure}
\centering
\includegraphics[width=\textwidth]{pre-of-timeseries_sgzmuvbase}
\caption{S-G,ZMUV confidence and success.}
\end{figure}




\begin{figure}
\centering
\includegraphics[width=\textwidth]{pre-of-timeseries_zmuvposeR}
\caption{ZMUV pose R.}
\end{figure}



\begin{figure}
\centering
\includegraphics[width=\textwidth]{pre-of-timeseries_sgzmuvposeR}
\caption{S-G,ZMUV pose R.}
\end{figure}


\begin{figure}
\centering
\includegraphics[width=\textwidth]{pre-of-timeseries_zmuvposeT}
\caption{ZMUV pose T.}
\end{figure}

\begin{figure}
\centering
\includegraphics[width=\textwidth]{pre-of-timeseries_sgzmuvposeT}
\caption{S-G,ZMUV pose T.}
\end{figure}








\end{document}
